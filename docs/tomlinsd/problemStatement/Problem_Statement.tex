\documentclass[onecolumn, draftclsnofoot,10pt, compsoc]{IEEEtran}
\usepackage{graphicx}
\usepackage{url}
\usepackage{setspace}

\usepackage{geometry}
\geometry{textheight=9.5in, textwidth=7in}

% 1. Fill in these details
\def \CapstoneTeamName{		We Heart Corvallis}
\def \CapstoneTeamNumber{		14}
\def \GroupMemberOne{			Dylan Tomlinson}

\def \CapstoneProjectName{		I Heart Corvallis}
\def \CapstoneSponsorCompany{	Corvallis Community Relations}
\def \CapstoneSponsorPerson{		Lyndi-Rae Elisabeth Petty}

% 2. Uncomment the appropriate line below so that the document type works
\def \DocType{		Problem Statement
				%Requirements Document
				%Technology Review
				%Design Document
				%Progress Report
				}
			
\newcommand{\NameSigPair}[1]{\par
\makebox[2.75in][r]{#1} \hfil 	\makebox[3.25in]{\makebox[2.25in]{\hrulefill} \hfill		\makebox[.75in]{\hrulefill}}
\par\vspace{-12pt} \textit{\tiny\noindent
\makebox[2.75in]{} \hfil		\makebox[3.25in]{\makebox[2.25in][r]{Signature} \hfill	\makebox[.75in][r]{Date}}}}
% 3. If the document is not to be signed, uncomment the RENEWcommand below
\renewcommand{\NameSigPair}[1]{#1}

%%%%%%%%%%%%%%%%%%%%%%%%%%%%%%%%%%%%%%%
\begin{document}
\begin{titlepage}
    \pagenumbering{gobble}
    \begin{singlespace}
    	% \includegraphics[height=4cm]{coe_v_spot1}
        \hfill 
        % 4. If you have a logo, use this includegraphics command to put it on the coversheet.
        %\includegraphics[height=4cm]{CompanyLogo}   
        \par\vspace{.2in}
        \centering
        \scshape{
            \huge CS Capstone \DocType \par
            {\large\today}\par
            \vspace{.5in}
            \textbf{\Huge\CapstoneProjectName}\par
            \vfill
            {\large Prepared for}\par
            \Huge \CapstoneSponsorCompany\par
            \vspace{5pt}
            {\Large\NameSigPair{\CapstoneSponsorPerson}\par}
            {\large Prepared by }\par
            Group\CapstoneTeamNumber\par
            % 5. comment out the line below this one if you do not wish to name your team
            \CapstoneTeamName\par 
            \vspace{5pt}
            {\Large
                \NameSigPair{\GroupMemberOne}\par
            %    \NameSigPair{\GroupMemberTwo}\par
            %   \NameSigPair{\GroupMemberThree}\par
            }
            \vspace{20pt}
        }
        \begin{abstract}
        % 6. Fill in your abstract    
            The purpose of this document is to define the problem that Team 14 will be attempting to solve throughout the
            next three terms of Senior Design. This document also contains a proposed solution for the problem that
            is defined. This document also has a list of performance metrics that will be used to determine the
            success of the project.
        \end{abstract}     
    \end{singlespace}
\end{titlepage}
\newpage
\pagenumbering{arabic}
\tableofcontents
% 7. uncomment this (if applicable). Consider adding a page break.
%\listoffigures
%\listoftables
\clearpage

% 8. now you write!
\section{Definition of Problem}

A current problem that the Office of Student Life (aka Corvallis Community Relations) is facing is that not enough
students are involving themselves with the Corvallis community. There may be several reasons for this problem.
One of them could be because students simply do not know about these events that are happening, or they just don't
have enough information about the events, such as the times or places they occur. Another reason could be that
students do not have a good enough reason to go to these events, they may not think that these events will benefit them.

\section{Proposed Solution}
Our solution is to create an application that will help advertise and give incentive for civic engagement among the
students of Corvallis. This application will be available on Android and iOS platforms to ensure the majority of students can make use of the application.
The application is intended to be used as a passport for the user. Information about local events will be posted on the application,
and the user can decide whether or not to attend this event. If the user does attend the event, they collect a
sort of stamp on their application. When a certain number of stamps are claimed, the user will receive a reward. This gives an incentive for the user to go to more events, which solves on of the stated problems.
Another benefit of this is that users will have the ability to see information about events and their locations. This will help users to be more informed,
about what goes on in the Corvallis community and will hopefully increase turnouts for these events. This also solves another of the problems
listed above.

\section{Performance Metrics}
\begin{itemize}
    \item A down-loadable application for  both Android and iOS that launches.
    \item GPS tracking to ensure users get stamps for the events they attend.
    \item A reward system that will give users rewards based on the events the events they attend.
    \item A system that allows our client to easily add and remove events, including information about the events, location, etc.
    \item A system that allows users to create a profile that includes basic information about them, a photo, etc.
    \item A system that will allow users to view events in the coming weeks or months, allowing them to select certain events and add them to their calendar.
    \item Ensure that the application can allocate at least 50 users at once.
    \item A database that is able to hold/release information on at least 50 user accounts
\end{itemize}

\end{document}