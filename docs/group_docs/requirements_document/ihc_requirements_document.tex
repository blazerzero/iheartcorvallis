\documentclass[draftclsnofoot, onecolumn, 10pt, compsoc]{IEEEtran}

\usepackage[english]{babel}
\usepackage{amsmath}
\usepackage{graphicx}
\usepackage[top=0.75in, bottom=0.75in, left=0.75in, right=0.75in]{geometry}

\title{\textbf{I Heart Corvallis - Mobile Application\\Requirements Document}\\Capstone I\\Fall 2017}

\author{Omeed Habibelahian\\Bradley Imai\\Dylan Tomlinson}

\begin{document}
	\maketitle
	\newpage
	
	\section{Introduction}
		\subsection{Purpose}
			This document will define the scope and requirements for completing the development of the I Heart Corvallis mobile application. It defines key terms central to the development of the application and its background, constraints and software complications that could influence completion of this project, and what key features will need to be implemented in the application before it can be considered completed. This document's intended audience is the Corvallis Community Relations office, who we are building this application for, and our instructors that will be grading and evaluating the project at the end of the year.
			
		\subsection{Scope}
			In this project, we will be producing the "I Heart Corvallis" mobile application. The app will showcase events happening around the Corvallis community, such as city council meetings, service and volunteer projects, and other community activities. It will also act as a passport for users to show that they have attended these activities. The app will give the user stamps upon completion or verification of attendance for each activity and will offer rewards to the user for accumulating enough stamps. On top of this, the application will showcase other resources available to community members. \\ \\
			The application will be available for Android and iOS devices and aims to inform members of the Corvallis community, both students and others, about various initiatives and resources around the community, as well as get community members more involved with community projects, events, and meetings by giving them an incentive to do so.
The application will essentially be a passport for students. They will receive stamps for helping out with service projects, attending city council meetings, attending workshops, and attending other community activities, and will be rewarded for accumulating enough stamps. \\ \\
			Another goal of the app is to help students be more aware of community events.  To accomplish this  the application will utilize the Google Maps API to show where events and various community resources can be found. The app will also include a separate page that will provide additional information about the city of Corvallis, such as links to website in the community and information about our office and the initiative.
			
		\subsection{Definitions, Acronyms, and Abbreviations}
			\subsubsection{I Heart Corvallis} The application being developed in this project.
			\subsubsection{Corvallis Community Relations} A subset of the Office of Student Life. The Corvallis Community Relations office is the leader of the I Heart Corvallis initiative and the client for this project.
			\subsubsection{Stamp} In-app verification that the user has attended a particular community activity. Some activities will be worth more stamps than others.
			\subsubsection{IDE} Integrated Development Environment; a software application that provides comprehensive facilities for software development. An IDE typically consists of a source code editor, build automation tools, and a debugger.
			\subsubsection{macOS} Apple's desktop operating system; formerly known as OS X.
			\subsubsection{iOS} Apple's mobile operating system, used on iPads, iPhones, and iPod touches.
			\subsubsection{Android} Google's open-source mobile operating system
			\subsubsection{Xcode} The official IDE for software development on Apple's mobile and desktop operating systems.
			\subsubsection{Android Studio} The official IDE for software development on Google's Android operating system, built on JetBrains's IntelliJ IDEA software and designed specifically for Android development.
			\subsubsection{Database} A structured set of data held in a computer, especially one that is accessible in various ways.
			\subsubsection{Node.js} A runtime environment used for executing server-side JavaScript vode.
			\subsubsection{Meteor} A web framework written in Node.js that allows for the production of cross-platform code between Android, iOS, and webpages.
			\subsubsection{MongoDB} A document-oriented database program.
					
		\subsection{Overview}
			The client of this application is the Corvallis Community Relations office. They have asked us to create a mobile application to aid them to list the various community events, projects, and meetings being put on by different organizations, authenticates the user, successfully tracks the events they engage in, and provides rewards for completing enough community activities. The I Heart Corvallis mobile application will be dedicated to OSU students and community members.  
	
	\section{Overall Description}
		\subsection{Product Perspective}
			This product will be visually similar to that of other Oregon State University mobile applications and desktop pages, but functionally it is completely independent of any other systems. The Corvallis Community Relations office wants to use the I Heart Corvallis application in conjunction with their Corvallis Living Guide website, but the two will not be directly reliant on each other.
		
		\subsection{Product Functions}
			When the user first opens the application, it will show the log-in page. This page asks if the user is an OSU student or a permanent resident. Students will be redirected to the ONID login screen for authentication, while permanent residents will be asked to log in with their username and password. They will be given the option to create a username and password if they have not previously done so. \\ \\
			If the user is logging in for the first time, the app will display a short survey with a few questions for users to answer before their home page loads. \\ \\
			Once the user has logged in, the app will display the home page, which provides quick links to the user's passport, the list of upcoming events, the stamp leaderboard, prize page, community resources page, and a page about the creators of the application and information about the I Heart Corvallis initiative. \\ \\
			The event page lists community events in chronological order. Non-timed, location-based events will be listed first, followed by the events with a specific date and time being listed afterwards. Each event's information box will show the event title, location, and date and time if applicable. The user will be able to press on an event's box to find out more information about that event. The detailed event information page will include a picture for the event, as well as the event title, location, date and time if applicable, description, and any relevant links provided by the event host. Timed events will be differentiated from general location-based activities by an icon or different background color in the event's information box. \\ \\
			The passport page shows stamps accumulated from attending community activities and will give the user the option to redeem prizes if they have accumulated enough stamps. \\ \\
			The community resources page will show other resources available to community members, as well as a link to the Corvallis Living Guide website. This page will also include a Google Map that will display pinpoint various community establishments such as entertainment locations, grocery stores, restaurants, shopping locations, and city offices. The "about" page will provide a short description of the app and its purpose, as well as information about the Corvallis Community Relations office. \\ \\
			The Corvallis Community Relations office will also have access to a back-end page that they will use to add, edit, and remove events, as all events will have to be approved by the office before being added to the application's event list, and the office will have the sole authority to post, edit, and remove these events.
		
		\subsection{User Characteristics}
			The intended users of the I Heart Corvallis app are OSU students and permanent residents of Corvallis. The app does not require any special technical or educational expertise, as it will be designed for anyone to be able to navigate without needed to learn any new technical skills. As long as the user knows how to use a mobile application, the app will be designed to be simple enough for even non-tech savvy users to navigate.
		
		\subsection{Constraints}
			There are a couple of constraints that can potentially limit our options as far as project completion is concerned. The Corvallis Community Relations office wants the application to be developed for both Android and iOS, but Android and iOS apps are usually coded in different languages and tested in different IDEs. For example, Android apps are generally coded in Java and XML and tested in Android Studio, while Apple pushes Objective-C and Swift as the ideal languages for coding iOS apps and offers Xcode as the official IDE for developing for iOS. Therefore, efficiently coding for Android and iOS simultaneously in a cross-platform fashion will be a challenge. Because of this, our primary focus will be on completing the Android version of the app, and if time allows after completing the Android version, we will build the iOS version. \\ \\
			Another constraint is the availability of testing environments across multiple desktop platforms. While Android Studio is available for both Mac and Windows, Xcode is a Mac-exclusive IDE, and not everyone on the development team has a Mac computer. Therefore, we will have to find the best way for all three of us to build the iOS version of the application.
		
		\subsection{Assumptions and Dependencies}
			The main factor that will influence the requirements for this project is whether or not we can successfully build the application in a cross-platform fashion. The Corvallis Community Relations office has mentioned Meteor, or MeteorJS, as an option for cross-platform implementation of the app. Meteor would allow us to build the application using JavaScript and would allow for synchronous cross-platform implementation across both Android and iOS without the use of either Android Studio or Xcode. However, Meteor uses MongoDB as its database, which we have historically found to be more difficult than other databases such as SQL. Meteor is also a platform that none of us on the development team have prior experience with, whereas we do have experience with IDEs like Android Studio.
		
	\section{Specific Requirements}
		\subsection{External Interface Requirements}
			\subsubsection{User Interfaces}
				When the application is opened for the first time, it will display a log-in screen. There will be two log-in systems implemented into the app, as the login for students will be separate than for permanent residents. For permanent residents, they will log in with a username and password, creating this username and password if they have not done so before, and this login information will be stored in a database that will hold all application information connected with that user's account. Students' account information will also be stored in this database, but we will instead implement the ONID log-in system for OSU students. \\ \\
				The app will display the home page whenever the user reopens the app. This page will display quick links to other pages, such as the user's passport, stamp leaderboard, prizes available, and a page about the creators of the application and the I Heart Corvallis initiative. \\ \\
				The passport system will allow the user to accumulate points, or "stamps," after completing or attending various community activities and events, and the user will be able to win prizes for accumulating enough stamps. \\ \\
				The application will also be connected to a database of events, as well as a database of user account information. The event database information will hold all of the information for each event being advertised by the app. The user account database will hold the user's login credentials, which will be kept secret to minimize the risk of the wrong user getting access to someone's account, as well as their passport information. \\ \\
				The application will provide a map that displays the locations of important and notable community resources, events, and activities, as well as sources of entertainment, restaurants, shopping centers, and city offices. We will utilize the Google Maps API to implement this map into the application. The locations of the events advertised in the app will be shown on the map as well, and the user will be able to click on a pin on the map for more information about the event, activity, or establishment that pin represents.
				
			\subsubsection{Hardware Interfaces}
				The only hardware restriction on the application is that it will initially be exclusively available on the Google Play Store for Android devices. Our stretch goal is to build an iOS version of the application and publish it to the App Store as well. Other than this, the application itself doesn't have any designated hardware interfaces.
			
			\subsubsection{Software Interfaces}
				The application will use the user's smartphone's GPS to track their current location to check if they are at the location of a community event or activity. It will also require an internet connection to retrieve event information from the event database and retrieve the user's information from the account database.
			
			\subsubsection{Communications Interfaces}
				Communication between different sections of the app is vital for the app's success, so links and buttons will have to take the user to the correct page. Links will redirect to the smartphone's Internet browser, pressing on an event box will bring up more information about that event, and quick links will redirect to their respective full pages. The underlying operating system will also be vital to successful communication between sections of the app.
			
		\subsection{Functional Requirements}
		
		\subsection{Performance Requirements}
			The application will support up to 20 simultaneous users. The user will not have to input very much information, as the app will generally show them upcoming community events and activities, how much of their passport they have completed, what prizes they can win and how they can redeem it, and various community resources available to them.
		
		\subsection{Design Constraints}
			Our main design constraints will come from whether or not we can successfully build the application in a cross-platform fashion. Building the application with cross-platform compatibility will decrease the amount of time it takes to complete development of the application, but building the app separately with their own native IDEs and in their own native environments could take quite a bit longer. This is why building the iOS version of the app is a stretch goal, as we would like to build the iOS version but realize that depending on whether or not we can successfully implement the app with cross-platform compatibility, it may not be as feasible in the time frame that we have for this project.
		
		\subsection{Software System Attributes}
			\subsubsection{Reliability}
				The application will have a user interface that 8 out of 10 users will be able to navigate without difficulty. The application will also function without crashing 95\% of the time.
			
			\subsubsection{Availability}
				The application will initially be available for Android, and if we can meet our stretch goal, it will be able for iOS upon launch as well. The application will require an Internet connection to function properly. If the Internet connection is lost or the app crashes, the app will recover any account and event information that had already been successfully stored in the database.
			
			%\subsubsection{Security}
			%	For OSU students, the authentication system will be the same as for other services that require an ONID login for authentication. The student will have to log in with their ONID username and password to be able to access the app.
			
			%\subsubsection{Maintainability}
			
			%\subsubsection{Portability}
		
		\subsection{Stretch Goals}
			Our main stretch goal is to build and publish an iOS version of the application, complete with all of the features available on the Android version of the application.
	
\end{document}
		