\documentclass[draftclsnofoot, onecolumn, 10pt, compsoc]{IEEEtran}

\usepackage[english]{babel}
\usepackage{amsmath}
\usepackage{graphicx}
\usepackage[top=0.75in, bottom=0.75in, left=0.75in, right=0.75in]{geometry}

\title{\textbf{I Heart Corvallis - Mobile Application\\Problem Statement}\\Capstone I\\Fall 2017}

\author{Omeed Habibelahian\\Bradley Imai\\Dylan Tomlinson}

\begin{document}
	\maketitle
	\begin{abstract}
		This document details the "I Heart Corvallis" mobile application, which will be used to get students and other 
		community members more involved in various events, meetings, and projects. In many cases, students choose 
		not to get involved in community activities because they do not feel connected to the Corvallis community. "I Heart 
		Corvallis" will aim to solve that problem by giving students and community members an incentive to get involved by 
		rewarding them for how much they contribute to the community. The final product should consist of an application 
		available for both Android and iOS that lists the various community events, projects, and meetings being put on by 
		different organizations, authenticates the user, successfully tracks the events they engage in, and provides rewards 
		for completing enough community activities.
	\end{abstract}
	\newpage
	
	\section{The Problem}
		A current problem that the Corvallis Community Relations office is facing is that college students are not getting 
		involved with the Corvallis community. There may be several reasons for this problem. One reason could be that 
		students are not interested enough to get involved in activities around the community, or they might not think the 
		events are worth attending. Students often do not feel a sense of belonging or responsibility to our community, and 
		this is largely because students are only here for a short period of time, so they mostly identify with being a part of the 
		OSU community rather than the Corvallis community as a whole. Another reason could be that they are not aware of 
		opportunities available to them. These activities may not be advertised well enough for students to easily find them.
	
	\section{The Solution}
		We plan to tackle this problem by creating the "I Heart Corvallis" mobile app. Inspired by the Bend Visitor and 
		Convention Bureau's app "Bend Ale Trail," I Heart Corvallis will be available for both Android and iOS and aims to 
		inform members of the Corvallis community, both students and others, about various initiatives and resources around 
		the community, as well as get community members more involved with community projects, events, and meetings by 
		giving them an incentive to do so. \\ \\			
		The application will essentially be a passport for students. They will receive stamps for helping out with service 
		projects, attending city council meetings, attending workshops, and attending other community activities, and will be 
		rewarded for accumulating enough stamps. The rewards will work in tiers; as you gain more stamps, you become 
		eligible for better rewards. \\ \\			
		To help students be more aware of community events, the application will also utilize the Google Maps API to show 
		where events and various community resources can be found. The app will also include a separate page that will 
		provide additional information about the city of Corvallis, such as links to website in the community and information 
		about our office and the initiative. \\ \\			
		Functionally, the app will authenticate the user to load their "passport" (the events they have attended and any stamps 
		they have collected). These stamps will be stored in a database so that whenever the user logs in, they will be able to 
		see the stamps they have accumulated. Whenever the user attends another event, they will earn another stamp, and 
		this stamp will be added to the database. Depending on which event the user attends and completes, the number of 
		stamps awarded to them will vary. For example, walking up Bald Hill would give you one stamp, whereas volunteering 
		at a blood drive or big community event can give up to 2-4 stamps. There will also be a database of rewards available, 
		and the user will win a reward for completing enough activities. In order to claim a reward the user will have to visit the 
		Corvallis Community Relations office and show our clients the stamps they have collected. \\ \\			
		Users will be required to either log in or create an account to use the app. For OSU students, they will not need to 
		make a new account, as they can simply log in with the ONID username and password. For non-OSU students (such 
		as permanent residents), they will need to make an account with a unique username and password, and they will log 
		in using this username/password combination. There will be separate login screens for students and permanent 
		residents. \\ \\			
		The application will also be able to confirm that the user actually attended this activity. The user will not be able to just 
		say they attended it and get a stamp for it. In order to block users from abusing the system to reap rewards, there will 
		be a code specific to each event that will be given to the host of said event. The host will then give out the code to the 
		users at the end of the event, and the users will input the code to acquire that stamp. For location based events, there 
		will be a flyer or sign put up at the location that will specify for users of the I Heart Corvallis application to check-in on 
		their phones to acquire the stamp. \\ \\			
		The application will be restricted when it comes to who can post new events. Only the Corvallis Community Relations 
		office will be able to add new events to the app and configure the list of events in the passport, as they will review and 
		approve every event prior to adding it to the app. Students and other community members will not be able to post 
		events to the application. If they have an event they would like to add, they must go to the Corvallis Community 
		Relations office to get approval, and even then, the Corvallis Community Relations office maintains sole authority to 
		add the event to the app and configure the event passport appropriately.
	
	\section{Performance Metrics}
		Upon completion of the project, the above features will be fully implemented into the application, students will be able 
		to navigate throughout the app without difficulty and without the app crashing, the stamp system will successfully add 
		the appropriate stamps into each user's passport, and the Corvallis Community Relations office will have the ability to 
		add and remove events into the database without crashes or overflows. \\ \\
		The application will also be designed for longevity. The Corvallis Community Relations office does not have much 
		technical prowess available to them. Therefore, we need to ensure that this app will remain stable after our departure, 
		and that the app is intuitive enough for non-technical clients to add events and navigate the back end of the app.
	
\end{document}
	