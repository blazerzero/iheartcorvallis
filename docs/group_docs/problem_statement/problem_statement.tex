\documentclass[draftclsnofoot, onecolumn, 10pt, compsoc]{IEEEtran}

\usepackage[english]{babel}
\usepackage{amsmath}
\usepackage{graphicx}
\usepackage[top=0.75in, bottom=0.75in, left=0.75in, right=0.75in]{geometry}

\title{\textbf{I Heart Corvallis - Mobile Application\\Problem Statement}\\Capstone I\\Fall 2017}

\author{Omeed Habibelahian\\Bradley Imai\\Dylan Tomlinson}

\date{\today}

\begin{document}
	\maketitle
	\begin{abstract}
		This document details the "I Heart Corvallis" mobile application, which will be used to get students and other 
		community members more involved in various events, meetings, and projects. In many cases students either choose 
		not to get involved in community activities or aren't aware of them. "I Heart Corvallis" will aim to solve that problem by 
		giving students and community members an incentive to get involved and rewarding them for how much they 
		contribute to the community. The final product should consist of an application available for both Android and iOS that 
		lists the various community events, projects, and meetings being put on by different organizations, authenticates the 
		user, successfully tracks the events they engage in, and provides rewards for completing enough community activities.
	\end{abstract}
	\newpage
	
	\section{The Problem}
		A current problem that the Office of Student Life is facing is that college students are not getting involved with the 
		Corvallis community. There may be several reasons for this problem. One reason could be that students are not 
		interested enough to get involved in activities around the community, or they might not think the events are worth 
		attending. Another reason could be that they aren't aware of opportunities available to them. These activities may not 
		be advertised well enough for students to easily find them. 
	
	\section{The Solution}
		We plan to tackle this problem by creating the "I Heart Corvallis" mobile app. Inspired by the Bend Visitor and 
		Convention Bureau's app "Bend Ale Trail," I Heart Corvallis will be available for both Android and iOS and aims to 
		inform members of the Corvallis community, both students and others, about various initiatives and resources around 
		the community, as well as get community members more involved with community projects, events, and meetings by 
		giving them an incentive to do so. \\ \\ 
		The application will essentially be a passport for students. They will receive stamps for helping out with service 
		projects, attending city council meetings, attending workshops, etc., and will be rewarded for accumulating enough 
		stamps. The rewards will work in tiers, as you gain more stamps you become eligible for better rewards. \\ \\
		To help students be more aware of community events, the application will also utilize a Google Maps API that will 
		show the locations of events that are happening as little icons. The user will be able to click these icons in order to 
		gather more information about the event.
	
	\section{Performance Metrics}
		Upon completion of the project, we'll have an application that's available on both Android and iOS that displays a 
		timeline of various community events/projects/meetings and allows the user to press on the activity to learn more 
		about it. On top of this, the app will authenticate the user to load their "passport" (the events they've attended and any 
		stamps they've collected). These stamps will be stored in a database so that whenever the user logs in, they will be 
		able to see the stamps they've accumulated. Whenever the user attends another event, they will earn another stamp, 
		and this stamp will be added to the database. Depending on which event the user attends and completes, will 
		determine the amount of stamps they will be rewarded with. For example walking up bald hill would give you one 
		stamp, whereas volunteering at a blood drive or big community event can give up to 2-4 stamps. There will also be a 
		database of rewards available, and the user will win a reward for completing enough activities. In order to claim a 
		reward the user will have to visit the Office of Student Life and show our clients the stamps they?ve collected. \\ \\
		Users will be required to either log in or create an account to use the app. For OSU students, they won?t need to make 
		a new account, as they can simply log in with the ONID username and password. For non-OSU students (i.e. 
		permanent residents), they will need to make an account with a unique username and password, and they will log in 
		using this username/password combination. There will be separate login screens for students and permanent 
		residents. \\ \\
		The application will also be able to confirm that the user actually attended this activity. The user won't be able to just 
		say they attended it and get a stamp for it. In order to block users from abusing the system to reap rewards, there will 
		be a code specific to each event that will be given to the host of said event. The host will then give out the code to the 
		users at the end of the event, and the users will input the code to acquire that stamp. For location based events, there 
		will be a flyer or sign put up at the location that will specify for users of the I Heart Corvallis application to check-in on 
		their phones to acquire the stamp. \\ \\
		The application will be restricted when it comes to who can post new events. Only the Office of Student Life will be 
		able to add new events to the app and configure the list of events in the passport, as they will review and approve 
		every event prior to adding it to the app. Students and other community members will not be able to post events to the 
		application. If they have an event they?d like to add, they must go through the Office of Student Life to get approval, 
		and even then, the Office of Student Life will be the ones to add the event to the app and configure the event passport 
		appropriately. \\ \\
		The application also needs to be designed for longevity. The Office of Student Life does not have much technical 
		prowess available to them. Therefore, we need to ensure that this app will remain stable after our departure, and that 
		the app is intuitive enough for non-technical clients to add events and navigate the back-end of the app. \\ \\
		The application will be complete if all of these features and conditions are included and met.

	
\end{document}
	