\documentclass[draftclsnofoot, onecolumn, 10pt, compsoc]{article}

\usepackage[english]{babel}
\usepackage{amsmath}
\usepackage{graphicx}
\usepackage[top=0.75in, bottom=0.75in, left=0.75in, right=0.75in]{geometry}
\setlength{\parindent}{15pt}

\title{\textbf{I Heart Corvallis Mobile Application\\Problem Statement}\\Capstone I\\CS 472\\Fall 2017}

\author{Omeed Habibelahian}

\date{\today}

\begin{document}
	\maketitle
	\begin{abstract}
		\paragraph{ }
		\indent{
		This document details the "I Heart Corvallis" mobile application, which will be used to get students and other
		community members more involved in various events, meetings, and projects. In many cases students either
		choose not to get involved in community activities or aren't aware of them. "I Heart Corvallis" will aim to solve
		that problem by giving students and community members an incentive to get involved and rewarding them for
		how much they contribute to the community. The final product should consist of an application available for both
		Android and iOS that lists the various community events, projects, and meetings being put on by different
		organizations, authenticates the user, successfully tracks the events they engage in, and provides rewards for
		accumulating enough "stamps" (this term will be further explained in the full document body).
		}
	\end{abstract}
	\newpage
	
	\section{The Problem}
		\paragraph{ }
		\indent{
		College students often either don't care enough to get involved in activities around the community, or they do
		want to but aren't aware of opportunities available to them. These activities may not be advertised well enough
		for students to easily find them.
		}
	
	\section{The Solution}
		\paragraph{ }
		\indent{
		We plan to tackle this problem by creating the "I Heart Corvallis" mobile app. The app will be available for both
		Android and iOS and aims to inform members of the Corvallis community, both students and others, about various
		initiatives and resources around the community, as well as get community members more involved with community 
		projects, events, and meetings by giving them an incentive to do so.
		}
		\paragraph{ }
		\indent{
		The application will essentially be a passport for students. They will get stamps for helping out with service projects,
		attending city council meetings, attending workshops, etc., and will be rewarded for accumulating enough stamps.
		}
	
	\section {Performance Metrics}
		\paragraph{ }
		\indent{
		Upon completion of the project, we'll have an application that's available on both Android and iOS that displays a
		timeline of various community events/projects/meetings and allows the user to press on the activity to learn more
		about it. On top of this, the app will authenticate the user to load their "passport" (the events they've attended and
		any stamps they've collected). These stamps will be stored in a database so that whenever the user logs in, they
		will be able to see the stamps they've accumulated. Whenever the user attends another event, they will earn
		another stamp, and this stamp will be added to the database. There will also be a database of rewards available,
		and the user will win a reward for completing enough activities.
		}
		\paragraph{ }
		\indent{
		The application will also be able to confirm that the user actually attended this activity. The user won't be able to just
		say they attended it and get a stamp for it. In order to block users from abusing the system to reap rewards, there will
		need be an authentication mechanism that confirms the user actually attended the activity, and the final version of the
		app will include this feature.
		}
		\paragraph{ }
		\indent{
		The application will be complete if all of these features and conditions are included and met.
		}
		
		\subsection {Note}
			\paragraph{ }
			\indent{
			We are having our first meeting with our client on Wednesday, October 12 and I will be able to more
			accurately portray the specifics of the application after that meeting.
			}
	
\end{document}
	
	