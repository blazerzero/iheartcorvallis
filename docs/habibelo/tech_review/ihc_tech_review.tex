\documentclass[draftclsnofoot, onecolumn, 10pt, compsoc]{IEEEtran}

\usepackage[english]{babel}
\usepackage{amsmath}
\usepackage{graphicx}
\graphicspath{	{./}	}
\usepackage[top=0.75in, bottom=0.75in, left=0.75in, right=0.75in]{geometry}

\title{\textbf{I Heart Corvallis - Mobile Application\\Technology Review}\\Capstone I\\Fall 2017}

\author{Omeed Habibelahian}

\begin{document}
	\maketitle
	\begin{abstract}
		%ABSTRACT
	\end{abstract}
	\newpage
	
	\section{Data Storage and Handling}
		\subsection{Overview}
			Data storage and communication between databases will be crucial in our application. The application will utilize at least three separate databases, one for event information, one for prize information, and at least one for user account information (depending on whether or not we make separate databases for students and permanent residents), so information will frequently be shared and synchronized across the databases.
		\subsection{Criteria}
			We will need the ability to modify database information as users interact with the app. On top of users needing to be able to view the event information, show interest in an event, view prizes, and be rewarded for attending events, the Corvallis Community Relations office needs to be able to edit the information in these databases. Any changes made to events by the office need to be synchronized across any other databases that also hold that information.
		\subsection{Choice 1: MySQL}
			MySQL is an open-source relational database management system written in C and C++. It works on many different system platforms, such as Linux, macOS, Windows, FreeBSD, and OpenBSD. Developers can also use PHP to enable access to MySQL databases in Android applications. MySQL is used in many well-known websites, like Facebook, Twitter, Flickr, and YouTube.
			~\cite{wiki:MySQL}
			~\cite{MySQL_Android}
			
		\subsection{Choice 2: MongoDB}
			MongoDB, classified as a NoSQL database program, is a free and open-source cross-platform document-oriented database program that uses JSON-like documents with schemas. MongoDB supports field and range queries, as well as regular expression searches. The queries can return specific fields of documents and also include user-defined JavaScript functions. MongoDB can also be integrated into Android applications via the MongoDB Stitch Android SDK.
			~\cite{wiki:MongoDB}
			~\cite{MongoDB_Android}
			
		\subsection{Choice 3: PostgreSQL}
			PostgreSQL, or simply Postgres, is an object-relational database management system with an emphasis on extensibility and standards compliance. It can handle workloads ranging from small single-machine applications to large Internet-facing application, and it is the default database on macOS. An Android application can communicate directly with a Postgres database by using the PostgreSQL JDBC driver.
			~\cite{wiki:PostgreSQL}
			~\cite{PostgreSQL_Android}
		
		\subsection{Discussing the Choices}
			All three database management systems can be used in Android applications. MongoDB supports the most programming languages at 27, followed by MySQL at 19 and Postgres at 9 languages. However, all three support both Java and PHP. Whereas MySQL and Postgres both provide JDBC drivers as a potential access method, MongoDB does not and only provides their proprietary protocol using JSON. MySQL is extremely popular, and as a result there are a lot of third-party applications, tools and integrated libraries which help greatly with many aspects of working with it. MySQL is also feature rich, fast, and offers some rather advanced security features. Postgres is not as popular as MySQL, but there are still many great third-party tools and libraries that are designed to make working with Postgres simple. Postgres is very powerful and as a result does a great job of handling many tasks very efficiently. However, it is so powerful that it can be overkill in some cases and may appear less performant than MySQL. It's not quite as recommended for speedy operations and simple setups.
			~\cite{DB_Comparison}
			~\cite{MySQL_vs_Postgres}
			
		\subsection{Conclusion}
			Although all three of these database management systems have their upsides, we will be using MySQL for our database implementation. MySQL provides great security features, and is fast, which is vital to our application due to the constant communication between the user and the database. Postgres is a good option but may be more powerful than we need for this application, and it's not quite as fast as MySQL. MongoDB has some great features, but we've tried implementing MongoDB as a database management system for a website in the past and spent weeks trying to set it up. We were never able to successfully implement MongoDB into our website and ended up switching to MySQL anyway.
	
	\section{UI Design Architectures}
		\subsection{Overview}
			The user interface is going to be a major part of the application, as people won't use the app if it doesn't look visually appealing. The app also has to be easy to use so users continue using the app. Because of this, it's critical to choose a UI architecture that makes navigating the app as simple and visually clean as possible.
		\subsection{Criteria}
			The user interface needs to include features like access to other pages, quick link buttons, and cards/boxes for events. Clutter needs to be kept to a minimum, and event listings need to be aligned with each other. The app interface needs to stand out and not look like a generic stock Android app. It also needs to incorporate OSU color schemes, fonts, and logos to remain consistent with the schemes of other OSU applications.
		\subsection{Choice 1: Android SDK}
			Android SDK is the official software developer kit for creating Android applications, provided by Google. It provides you with the API libraries and developer tools necessary to build, test, and debug apps for Android. It provides you with Android Studio, which is the official integrated developing environment (IDE) for Android and also allows you to emulate Android devices on your computer to see how your app would function on an actual Android device. Android's default programming language is Java, and its default formatting language is XML.
			~\cite{AndroidSDK_vs_SemanticUI}
			
		\subsection{Choice 2: Apache Cordova}
			Apache Cordova is an open-source mobile development framework that allows you to use HTML5, CSS3, and JavaScript for cross-platform development. Applications execute within wrappers targeted to each platform.
			~\cite{Apache_Cordova}
		
		\subsection{Choice 3: Semantic UI}
			Semantic UI is a UI component library implemented using a set of specifications designed around natural language. It uses HTML and CSS and provides tons of templates for cards, boxes, buttons, menus, and icons. 
			~\cite{Semantic_UI}
		
		\subsection{Discussing the Choices}
			Android SDK is the official SDK provided by Google and provides Android Studio, which is really helpful because it's a one-stop shop for building, testing, debugging, and emulating Android apps. It's also pretty easy to use. Using Android SDK, apps are coded in XML and Java, which we have prior experience with. However, unless you do some thorough design formatting, the end product will utilize the stock Android design style, which is fine if not basic. We want to build an app that is more consistent with the design of other OSU applications, and we do not want our app to look like every other stock Android app. Apache Cordova uses HTML5, CSS3, and JavaScript, which allows for quite a bit of control over how the interface looks. XML allows for this too, though. Semantic UI also utilizes HTML and CSS, but it also has the big upside of providing tons of templates for cards, boxes, menus, and buttons. It also provides a large library of icons, which will come in great handy throughout our application.
		
		\subsection{Conclusion}
			Although all three options definitely can come in handy in this program, we will be using Semantic UI along with the Android Studio IDE. Although Android's default languages are Java and XML, you can edit the XML file corresponding to each code file to recognize JavaScript as the language for that file. Using Android Studio will allow us to easily test our app, and Semantic UI will allow us to implement a clean interface without resorting to the stock Android style.
			~\cite{SemanticUI_JS}
		
	\section{User Experience Design}
		\subsection{Overview}
		\subsection{Criteria}
		\subsection{Choice 1: }
		\subsection{Choice 2: }
		\subsection{Choice 3: }
		\subsection{Discussing the Choices}
		\subsection{Conclusion}
		
	\bibliography{ihc_tech_review}
	\bibliographystyle{IEEEtran}
	
\end{document}